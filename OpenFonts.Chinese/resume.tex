%%%%%%%%%%%%%%%%%%%%%%%%%%%%%%%%%%%%%%%
% Deedy - One Page Two Column Resume
% LaTeX Template
% Version 1.2 (16/9/2014)
%
% Original author:
% Debarghya Das (http://debarghyadas.com)
%
% Original repository:
% https://github.com/deedydas/Deedy-Resume
%
% IMPORTANT: THIS TEMPLATE NEEDS TO BE COMPILED WITH XeLaTeX
%
% This template uses several fonts not included with Windows/Linux by
% default. If you get compilation errors saying a font is missing, find the line
% on which the font is used and either change it to a font included with your
% operating system or comment the line out to use the default font.
% 
%%%%%%%%%%%%%%%%%%%%%%%%%%%%%%%%%%%%%%
% 
% TODO:
% 1. Integrate biber/bibtex for article citation under publications.
% 2. Figure out a smoother way for the document to flow onto the next page.
% 3. Add styling information for a "Projects/Hacks" section.
% 4. Add location/address information
% 5. Merge OpenFont and MacFonts as a single sty with options.
% 
%%%%%%%%%%%%%%%%%%%%%%%%%%%%%%%%%%%%%%
%
% CHANGELOG:
% v1.1:
% 1. Fixed several compilation bugs with \renewcommand
% 2. Got Open-source fonts (Windows/Linux support)
% 3. Added Last Updated
% 4. Move Title styling into .sty
% 5. Commented .sty file.
%
%%%%%%%%%%%%%%%%%%%%%%%%%%%%%%%%%%%%%%%
%
% Known Issues:
% 1. Overflows onto second page if any column's contents are more than the
% vertical limit
% 2. Hacky space on the first bullet point on the second column.
%
%%%%%%%%%%%%%%%%%%%%%%%%%%%%%%%%%%%%%%


\documentclass[]{deedy-resume-openfont}
\usepackage{fancyhdr}
    
\pagestyle{fancy}
\fancyhf{}
    
\begin{document}

%%%%%%%%%%%%%%%%%%%%%%%%%%%%%%%%%%%%%%
%
%     LAST UPDATED DATE
%
%%%%%%%%%%%%%%%%%%%%%%%%%%%%%%%%%%%%%%
\lastupdated

%%%%%%%%%%%%%%%%%%%%%%%%%%%%%%%%%%%%%%
%
%     TITLE NAME
%
%%%%%%%%%%%%%%%%%%%%%%%%%%%%%%%%%%%%%%
\namesection{冯}{子扬}{ \urlstyle{same}\href{mailto:fuujiro@qq.com}{fuujiro@qq.com} | +86 155-2489-2259
}

%%%%%%%%%%%%%%%%%%%%%%%%%%%%%%%%%%%%%%
%
%     COLUMN ONE
%
%%%%%%%%%%%%%%%%%%%%%%%%%%%%%%%%%%%%%%

\begin{minipage}[t]{0.25\textwidth} 

%%%%%%%%%%%%%%%%%%%%%%%%%%%%%%%%%%%%%%
%     EDUCATION
%%%%%%%%%%%%%%%%%%%%%%%%%%%%%%%%%%%%%%

\section{教育经历} 
\sectionsep

\subsection{大连理工大学}
\descript{本科:计算机科学与技术}
\descript{电气信息类创新实验班}
\descript{GPA: 85.2/100}
\location{2016.09-2020.07}
\sectionsep

%%%%%%%%%%%%%%%%%%%%%%%%%%%%%%%%%%%%%%
%     LINKS
%%%%%%%%%%%%%%%%%%%%%%%%%%%%%%%%%%%%%%

\section{链接}
\sectionsep
Blog://  \href{https://blog.fuujiro.com/}{\bf blog.fuujiro.com} \\  
Github:// \href{https://github.com/fuujiro}{\bf @fuujiro} \\
LinkedIn://  \href{https://www.linkedin.com/in/fuujiro}{\bf fuujiro} \\

%%%%%%%%%%%%%%%%%%%%%%%%%%%%%%%%%%%%%%
%     Intro
%%%%%%%%%%%%%%%%%%%%%%%%%%%%%%%%%%%%%%

\section{自我介绍}
\subsection{应聘 \  后端开发实习生}
我热爱编程 \\
熟悉\textbf{C/C++}语言 \\
理解并使用\textbf{Linux/Unix}系统 \\
掌握\textbf{Redis}及数据库原理 \\
热衷于学习新技术 \\
对开源项目充满极大热情 \\
希望成为一名优秀的软件工程师 \\
\sectionsep

%%%%%%%%%%%%%%%%%%%%%%%%%%%%%%%%%%%%%%
%     SKILLS
%%%%%%%%%%%%%%%%%%%%%%%%%%%%%%%%%%%%%%

\section{技能}
\sectionsep
\subsection{编程}
\location{编程语言}
C++ \textbullet{} C \textbullet{} Python \textbullet{} \LaTeX \\
\location{操作系统}
Linux/Unix \textbullet{} Shell \textbullet{} Vim \\
\location{数据库}
Redis \textbullet{} SQL \\ 
\sectionsep

\subsection{语言}
普通话 - 母语 \textbullet{} 英语 - 四六级\\
\sectionsep

%%%%%%%%%%%%%%%%%%%%%%%%%%%%%%%%%%%%%%
%
%     COLUMN TWO
%
%%%%%%%%%%%%%%%%%%%%%%%%%%%%%%%%%%%%%%

\end{minipage} 
\hfill
\begin{minipage}[t]{0.73\textwidth} 

%%%%%%%%%%%%%%%%%%%%%%%%%%%%%%%%%%%%%%
%     Project
%%%%%%%%%%%%%%%%%%%%%%%%%%%%%%%%%%%%%%

\section{个人项目}
\sectionsep
\runsubsection{\href{https://github.com/fuujiro/JiroOS}{\bf JiroOS}}
\descript{A toy Operating System / 小型操作系统}
\location{2019.3}
\vspace{\topsep}
\begin{tightemize}
    \item 基于 X86 的小型操作系统实践,支持物理机或bochs等虚拟机运行
    \item 实现段页内存分配,进程线程管理,同步异步互斥,进程通信和文件系统
    \item 调研其他开源实现,正在实现 64 位的扩展和 ARM 架构的适配
    \end{tightemize}
\sectionsep

\runsubsection{\href{https://github.com/fuujiro/socket}{\bf Socket}}
\descript{Linux网络编程 / socket编程 / 聊天室}
\location{2018.12}
\begin{tightemize}
    \item 实现TCP / UDP的socket网络套接字
    \item 实现select/poll/epoll的IO多路复用
    \item 基于select实现了最简单的聊天室
    \end{tightemize}
\sectionsep

\runsubsection{\href{https://github.com/fuujiro/LIMS}{\bf LIMS}}
\descript{Library Information Management System / 图书管理系统}
\location{2017.6}
\begin{tightemize}
    \item 具有图书添加,删除,三路查询,畅销推荐功能
    \item 具有管理员,用户两种关系,带有账户找回功能,彩色交互终端界面
    \item 在 GitHub 上获得 \textbf{42 stars},是\textbf{图书管理系统} $^{C语言}$ 下star最多的项目
    \end{tightemize}
\sectionsep

%%%%%%%%%%%%%%%%%%%%%%%%%%%%%%%%%%%%%%
%     EXPERIENCE
%%%%%%%%%%%%%%%%%%%%%%%%%%%%%%%%%%%%%%

\section{在校经历}
\sectionsep
\runsubsection{国家级大学生创新创业项目}
\descript{机器人手臂视觉标定}
\location{2018.01 - Present | 中国·大连}
\begin{tightemize}
    \item 基于张正友标定算法的机器手臂标定研究
    \item 尝试ARUco标记的新标定板,对原有的标定算法进行优化,提出新算法
    \item 我们的 \href{https://github.com/DUT-RVLab}{\bf DUT-RVLab} 在 GitHub 上更新研究进度
    \end{tightemize}
\sectionsep

\runsubsection{卡迪夫大学(Cardiff Univ.)}
\descript{Big Data \& HPC Summer School}
\location{2018.07 - 2018.08 | Cardiff, UK}
\begin{tightemize}
    \item 参加由校方资助的与卡迪夫大学School of Engineering的暑期交流项目
    \item 学习大数据,机器学习和高性能计算相关的知识
    \item 成功完成了最后的Hackathon结课项目,实现了一个手势识别的小型Demo
    \end{tightemize}
\sectionsep

\runsubsection{WeGeek 微信小程序黑客马拉松大赛}
\descript{“时间叉叉”小程序}
\location{2018.12 | 中国·北京}
\begin{tightemize}
    \item 参加由\textbf{Segmentfault}和\textbf{微信}联合举办的微信小程序黑客马拉松
    \item 开发了一个尝试取代“腾讯文档”的省时组队的小程序
    \end{tightemize}
\sectionsep

%%%%%%%%%%%%%%%%%%%%%%%%%%%%%%%%%%%%%%
%     EXPERIENCE
%%%%%%%%%%%%%%%%%%%%%%%%%%%%%%%%%%%%%%



%%%%%%%%%%%%%%%%%%%%%%%%%%%%%%%%%%%%%%
%     OPEN SOURCE
%%%%%%%%%%%%%%%%%%%%%%%%%%%%%%%%%%%%%%

\section{开源贡献}
\begin{tabular}{ll}
\href{https://github.com/chyyuu/ucore_os_lab}{\bf chyyuu/ucore\_os\_lab} & 提交labanswer,增加对IO多路复用和x86的扩展支持 \\
\end{tabular}
\sectionsep

%%%%%%%%%%%%%%%%%%%%%%%%%%%%%%%%%%%%%%
%     AWARDS
%%%%%%%%%%%%%%%%%%%%%%%%%%%%%%%%%%%%%%

\section{所获奖项} 
\begin{tabular}{rll}
2018	     & 一等奖  & 辽宁省大学生计算机多媒体设计大赛 + 奖学金 \\
2018         & 二等奖  & 2018年第八届APMCM亚太地区大学生数学建模竞赛 \\
2017	     & 一等奖  & 2017年大连理工大学“Ti”杯电子设计大赛 \\
2017	     & 二等奖  & 大连市第二十六届大学生数学竞赛 \\
\end{tabular}
\sectionsep

%%%%%%%%%%%%%%%%%%%%%%%%%%%%%%%%%%%%%%
%     PUBLICATIONS
%%%%%%%%%%%%%%%%%%%%%%%%%%%%%%%%%%%%%%

% \section{Publications} 
% \renewcommand\refname{\vskip -1.5cm} % Couldn't get this working from the .cls file
% \bibliographystyle{abbrv}
% \bibliography{publications}
% \nocite{*}

\end{minipage} 
\end{document}  \documentclass[]{article}
